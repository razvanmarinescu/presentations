\documentclass[8pt,xcolor=table]{beamer}

\usepackage{graphicx}
\usepackage{caption}
\usepackage{subcaption}
\usepackage{transparent}
\usepackage{epstopdf} %converting to PDF
\usepackage{multicol} 
\usepackage{animate}[2017/05/18]

% \usepackage{pdfx}
 
% \usepackage[utf8]{inputenc}
% \usepackage[T1]{fontenc}
\usepackage[table]{xcolor}    % loads also »colortbl« 
%  \usepackage{enumitem}
% \usepackage{ucltemplate}
\usepackage{color}

\usepackage{comment}

\usepackage{tabularx} % make width of table columns evenly distributed (see http://tex.stackexchange.com/questions/60601/evenly-distributing-column-widths)
% \newcolumntype{Y}{>{\centering\arraybackslash}X}

% make entire row bold or italic in table
\newcommand\setrow[1]{\gdef\rowmac{#1}#1\ignorespaces}
\newcommand\clearrow{\global\let\rowmac\relax}
\clearrow


\usepackage{amssymb}% http://ctan.org/pkg/amssymb
\usepackage{pifont}% http://ctan.org/pkg/pifont
\newcommand{\cmark}{\ding{51}}%
\newcommand{\xmark}{\ding{55}}%


%\usepackage{pgfgantt} % for grantt charts
\usepackage{rotating}
\usepackage[graphicx]{realboxes}
\usepackage[export]{adjustbox}
\usepackage{array}

\usepackage{rotating}
% \usepackage{tabularx, booktabs} % make width of table columns evenly distributed (see http://tex.stackexchange.com/questions/60601/evenly-distributing-column-widths)
% \newcolumntype{Y}{>{\centering\arraybackslash}X}

\DeclareMathOperator*{\argmin}{arg\,min}
\DeclareMathOperator*{\argmax}{arg\,max}

\usepackage{tikz}
\usetikzlibrary{arrows,positioning, shapes.symbols,shapes.callouts,patterns,shapes,chains,calc,backgrounds,fadings}

% \definecolor{parCol}{rgb}{0.1, 0.1, 1}
% \definecolor{stCol}{rgb}{0.1, 0.6, 0.1}
% \definecolor{bothCol}{rgb}{0, 0.5, 0.5}

\definecolor{parCol}{rgb}{0, 0, 0}
\definecolor{stCol}{rgb}{0, 0, 0}
\definecolor{bothCol}{rgb}{0, 0, 0}
\definecolor{blue3}{HTML}{86B7FC} % med blue
\definecolor{blue1}{HTML}{B5F1FF} % light blue
\definecolor{blue2}{HTML}{E0F9FF} % very light blue

\newcolumntype{C}[1]{>{\centering\let\newline\\\arraybackslash\hspace{0pt}}m{#1}}

\setlength{\tabcolsep}{0.2em}

 
 %% OVERVIEW OF WORK SO FAR %%
 
%Information to be included in the title page:
\title{Modelling the Neuroanatomical Progression of Alzheimer's Disease and Posterior Cortical Atrophy}
\author[Raz]{
R\u{a}zvan V. Marinescu\vspace{1em} \newline \and \small{Supervisors: Polina Golland (current), Daniel C. Alexander (previous)}}

\institute{\small{Medical Vision Group, Massachusetts Institute of Technology}

\vspace{0em}
\small{Centre for Medical Image Computing, University College London, UK}
}

\date{}

% logo of my university
\titlegraphic{
   \begin{figure}
%    \begin{subfigure}{0.32\textwidth}
%    \hspace{2em}
%    \includegraphics[height=1.0cm]{ucl_logo}
%    \end{subfigure}
   \begin{subfigure}{0.32\textwidth}
   \centering
   \includegraphics[height=1.0cm]{MIT_logo.png} 
   \end{subfigure}
%    \begin{subfigure}{0.32\textwidth}
%    \centering
%    \includegraphics[height=1.0cm]{pondLogo.png} 
%    \end{subfigure}
   \end{figure}
   
   \tiny{Slides available online: https://people.csail.mit.edu/razvan/talk/martinos2019/pres.pdf}
}

\setbeamercolor{frametitle}{fg=black}
\setbeamercolor{author in head/foot}{fg=black, bg=white} 
\setbeamercolor{institute in head/foot}{fg=black, bg=white} 
\setbeamercolor{title in head/foot}{fg=black, bg=white}
\setbeamercolor{date in head/foot}{fg=black, bg=white}

\setbeamersize{text margin left=10pt,text margin right=10pt}
% \setbeamertemplate{frametitle}{
%     \vspace{0.9em}
%     \insertframetitle
% %     \vspace{-3em}
% }
\setbeamertemplate{frametitle}{%
    \vspace{0.5em}
    \usebeamerfont{frametitle}\insertframetitle%
    \vphantom{g}% To avoid fluctuations per frame
    %\hrule% Uncomment to see desired effect, without a full-width hrule
    \par% <-- added
    \hspace*{-\dimexpr0.5\paperwidth-0.5\textwidth}% <-- calculation of left margin width
    \rule[0.5\baselineskip]{\paperwidth}{0.4pt}%
}

\setbeamertemplate{footline}
{
  \vspace{-3em}
  \leavevmode%
   \rule{\paperwidth}{0.3pt}
  \hbox{%
  \begin{beamercolorbox}[wd=.2\paperwidth,ht=2.25ex,dp=1ex,center]{author in head/foot}%
    \usebeamerfont{author in head/foot}Razvan V. Marinescu
  \end{beamercolorbox}%
  \begin{beamercolorbox}[wd=.2\paperwidth,ht=2.25ex,dp=1ex,center]{institute in head/foot}%
    \usebeamerfont{institute in head/foot}razvan@csail.mit.edu
  \end{beamercolorbox}%
  \begin{beamercolorbox}[wd=.3\paperwidth,ht=2.25ex,dp=1ex,center]{institute in head/foot}%
    \usebeamerfont{institute in head/foot}https://people.csail.mit.edu/razvan/
  \end{beamercolorbox}%
  \begin{beamercolorbox}[wd=.2\paperwidth,ht=2.25ex,dp=1ex,center]{title in head/foot}%
    \usebeamerfont{title in head/foot}\insertsection
  \end{beamercolorbox}%
  \begin{beamercolorbox}[wd=.10\paperwidth,ht=2.25ex,dp=1ex,right]{date in head/foot}%
    \usebeamerfont{date in head/foot}\insertshortdate{}\hspace*{2em}
    \insertframenumber{} / \inserttotalframenumber\hspace*{2ex}
  \end{beamercolorbox}}%
  \vskip0pt%
}

% \usepackage{beamerthemesplit}

\newcommand{\backupbegin}{
   \newcounter{finalframe}
   \setcounter{finalframe}{\value{framenumber}}
}
\newcommand{\backupend}{
   \setcounter{framenumber}{\value{finalframe}}
}


\makeatletter
\long\def\beamer@author[#1]#2{%
  \def\and{\tabularnewline}
  \def\insertauthor{\def\inst{\beamer@insttitle}\def\and{\tabularnewline}%
  \begin{tabular}{rl}#2\end{tabular}}%
  \def\beamer@shortauthor{#1}%
  \ifbeamer@autopdfinfo%
    \def\beamer@andstripped{}%
    \beamer@stripands#1 \and\relax
    {\let\inst=\@gobble\let\thanks=\@gobble\def\and{, }\hypersetup{pdfauthor={\beamer@andstripped}}}
  \fi%
}
\makeatother
\beamertemplatenavigationsymbolsempty
\setbeamertemplate{caption}[numbered]
\setbeamercolor{caption name}{fg=black}
\setbeamercolor{itemize item}{fg=black}
\setbeamercolor{itemize subitem}{fg=black}
\setbeamercolor{enumerate item}{fg=black}
\setbeamercolor{enumerate subitem}{fg=black}
\setbeamertemplate{enumerate item}[default]
\setbeamertemplate{enumerate subitem}[default]

\makeatletter
\let\@@magyar@captionfix\relax
\makeatother
\begin{document}
 
\section{Introduction}

\frame{\titlepage}
 
\setbeamerfont{frametitle}{size=\large}

\newcommand{\upgradeReportLoc}{../../upgrade_report}
\newcommand{\epsrcPresLoc}{\upgradeReportLoc/epsrcPres}
\newcommand{\jointModellingDiseaseLoc}{../../jointModellingDisease}
\newcommand{\pcaLongPaperLoc}{../../PCA_long_paper}
\newcommand{\voxFld}{../../voxelwiseDPM}
\newcommand{\tadpoleFld}{../../tadpole}
\newcommand{\diffEqModelFld}{../../diffEqModel}



\newcommand*{\pcaLongFigs}{\pcaLongPaperLoc/figures}


% \includeonlyframes{1-20}
%\includeonlyframes{current}



\newcommand{\ovHeight}{2cm}


% % TODO continue with overview, move into commands
\newcommand{\ovEBM}{
\begin{subfigure}{0.47\textwidth}
\centering
1. Modelled progression of PCA and tAD\\
(using existing methods)
\includegraphics[height=\ovHeight]{ebm_thumb.png}
\end{subfigure}
}

\newcommand{\ovVWDPM}{
\begin{subfigure}{0.47\textwidth}
\centering
% \vspace{2.8em}
2. Developed Novel Spatio-temporal Model \\ (DIVE)\\
\includegraphics[height=\ovHeight]{\upgradeReportLoc/images/vwdpm/blend14_adniThavgFWHM0InithistCl3Pr0Ra1_VWDPMStd.png}
\end{subfigure}
}


\newcommand{\ovDKT}{
\begin{subfigure}{0.47\textwidth}
\centering
\vspace{2em}
3. Developed Novel Transfer Learning \\ method (DKT) \\
\vspace{0.5em}
\includegraphics[height=2.2cm]{\jointModellingDiseaseLoc/paper/figures/disease_knowledge_transfer.pdf}
\end{subfigure}
}


\newcommand{\ovTadpole}{
\begin{subfigure}{0.47\textwidth}
\centering
\vspace{-2em}
4. Organised TADPOLE Competition\\
\vspace{1em}
\includegraphics[height=1.2cm,valign=t]{\upgradeReportLoc/epsrcPres/tadpole} 
\end{subfigure}
}

\newcommand{\ovPainter}{
\begin{subfigure}{\textwidth}
\centering
\vspace{0.5em}
5. Created BrainPainter software\\
\includegraphics[height=1.5cm]{cortical-front_1}\includegraphics[height=1.5cm]{cortical-back_1}\includegraphics[height=1.5cm]{subcortical_1}
\end{subfigure}
}


\definecolor{light-gray}{gray}{0.6}



% \begin{comment}

% \begin{frame}
% \frametitle{About me}
% 
% \begin{itemize}
%  \item Grew up in Pitesti, Romania
%   \item 2010-2014: Studied a 4-year MEng in Computer Science at Imperial College London
%   \item 2014-2019: PhD in Medical Imaging at UCL (with Daniel Alexander)
%   \item 2019: Postdoc at MIT with Pollina Golland (working on image analysis of stroke)
% 
%   \begin{figure}
%   \vspace{1em}
%   \includegraphics[height=2.5cm]{pitestiRomania}\hspace{1em}\includegraphics[height=2.5cm,trim=0 0 0 100,clip]{uclFrontEng.jpg} \hspace{1em}
%   \end{figure}
%  
%  \vspace{1em}
% 
%  
%  
%   \begin{figure}
%    \centering
% %     \hfill
%     \includegraphics[height=2.5cm]{Danny-Alexander.jpeg}
%     \hspace{3em}
%     \includegraphics[height=2.5cm]{polina}  
% %     \hfill
%   \end{figure}
% 
% 
% 
% \end{itemize}
% 
% \end{frame}

\begin{frame}{asdsda}
 
 
\end{frame}



\begin{frame}
\frametitle{Alzheimer's Disease is a Devastating Disease}

\vspace{-1em}
\begin{itemize}
 \item 46 million people affected worldwide
 
  \begin{figure}
 \centering
%   \includegraphics[height=3cm]{adPrelavence}
  \includegraphics[height=4cm]{adPrevalanceIncreasing}
 \end{figure}
 
 \onslide<2-> \item No treatments available that stop or slow down cognitive decline
 \onslide<2-> \item Q: Why did clinical trials fail? A: Treatments were not administered early enough 
 \vspace{1em}
 \onslide<3-> \item Q: How can we then identify subjects \textbf{early} in order to administer treatments? 
 \onslide<3-> \item A: Biomarkers ...
 


 

\end{itemize}

\vspace{-1em}

\end{frame}

% Also say that we cannot build the model on age
\begin{frame}
\frametitle{Biomarker Evolution creates a Unique Disease Signature\\
that can be used for Staging Individuals in Clinical Trials}
% explain what are the challenges

\begin{figure}
\centering 
\vspace{1em}
\includegraphics[height=5cm]{adniDiseaseProgression}
\hspace{-4em}ADNI website 

\end{figure}


\begin{itemize}
 \item Accurate disease staging $\rightarrow$ better patient stratification
 \item Problem: This is a "hypothetical" (i.e. qualitative) disease progression model
 \item Why construct a quantitative model? 
\end{itemize}

\end{frame}


\section{Disease Progression Modelling}



\begin{frame}
\frametitle{Benefits of Quantitative Disease Progression Models}

\begin{overprint}
 \onslide<1>\begin{figure}
 \centering
\includegraphics[height=5cm,trim=0 0 650 0,clip]{dpmDiffDiag1.png}
\end{figure}

\onslide<2> \begin{figure}
 \centering
\includegraphics[height=5cm,trim=0 0 650 0,clip]{dpmDiffDiag2.png}
\end{figure}

\onslide<3-> \begin{figure}
 \centering
\includegraphics[height=5cm,trim=0 0 0 0,clip]{dpmDiffDiag2.png}
\end{figure}

\end{overprint}

\vspace{1em}
\begin{itemize}
 \onslide<1-> \item Basic biological insight
 \onslide<2-> \item Staging can help stratification in clinical trials
 \onslide<3-> \item Differential diagnosis and prognosis
 
 \vspace{1.5em}
 \onslide<4-> \item[] How can we build such a disease progression model?
\end{itemize}

\end{frame}




\begin{frame}[label=current]
\frametitle{My PhD Contributions}

\begin{figure}
\centering


\ovEBM
\ovVWDPM

\ovDKT
\ovTadpole

\ovPainter

\end{figure}

\end{frame}




\section{DIVE}

\begin{frame}
\frametitle{My PhD Contributions}

%% new slide

\begin{figure}
\centering


{\transparent{0.4} 
\ovEBM 
}
\ovVWDPM

{\transparent{0.4} 
\ovDKT
\ovTadpole

\ovPainter
}

\end{figure}

\end{frame}


\begin{frame}
\frametitle{Aim: Build a Disease Progression Model of Pathology over the Brain that Avoids Limitations of Previous Models}


\newcommand{\aimImgScale}{0.8}
\newcommand{\mnpHeight}{3cm}

\vspace{-2em}

% FWHM0 avg thickness map MCI & AD
\begin{figure}[h]
  \centering
  \begin{minipage}[t][\mnpHeight][t]{0.3\textwidth}
   \centering
   Avoids pre-defined ROI parcellation\\
  \begin{tikzpicture}[scale=1]
    \node[inner sep=0] (image) at (0,0) {\includegraphics[width=0.5\textwidth,trim=0 0 26 0,clip]{seeley_2009_topleft.png}}; 
  \end{tikzpicture}
%     \caption{}
%       \label{fig:adniClust}
  \end{minipage}
   \begin{minipage}[t][\mnpHeight][t]{0.3\textwidth}
  \centering
  Avoids simplistic spatial correlation structure\\
%   \vspace{1em}
  \begin{tikzpicture}[scale=1]
    \node[inner sep=0] (corr_text) at (0,0) {\includegraphics[width=\aimImgScale\textwidth, trim=0 0 360 0, clip]{bilgel_neuroimage}};
  \end{tikzpicture}
  \vspace{0.5em}
  \end{minipage}
   \begin{minipage}[t][\mnpHeight][t]{0.3\textwidth}
  \centering
  Avoids simplistic biomarker trajectories
  \begin{tikzpicture}[scale=1]
    \node[inner sep=0] (corr_text) at (0,0) {\includegraphics[width=\aimImgScale\textwidth,trim=0 0 0 30,clip]{biomkStepFunctions}};
  \end{tikzpicture}
  \end{minipage}
\end{figure}

\vfill

This leads to a technique that simultaneously:
\begin{itemize}
 \item parcellates the brain into disconnected components that undergo similar progression
 \item estimates biomarker trajectories
\end{itemize}


\end{frame}


\begin{frame}
\frametitle{Motivation: Correlate with brain networks + better prediction/staging}


\begin{itemize}
 \item \textbf{Aim}: Move from ROI-based analysis to voxelwise/vertexwise
\end{itemize}

\begin{figure}
 \centering
  \begin{tikzpicture}[scale=1]
     \node (roi) at (0,0) {\includegraphics[scale=0.10]{clust24_drcThFWHM0InitfsurfCl4Pr0Ra1Mrf5_VWDPMStaticPCA.png}};
     \node (vw) at (4,0) {\includegraphics[scale=0.10]{clust24_drcThFWHM0Initk-meansCl4Pr0Ra1Mrf5_VDPM_MRFPCA.png}};
     \draw[line width=1.5,->] (roi) -> (vw);
  \end{tikzpicture}
\end{figure}



\begin{figure}
\begin{subfigure}{0.48\textwidth}
%\textbf{Motivation}:
\begin{enumerate}
\item Atrophy correlates with functional networks, which are not spatially connected (Seeley et al., Neuron, 2009)
\vspace{2em}
\item Better biomarker prediction and disease staging
\end{enumerate}
\end{subfigure}
% \hspace{1em}
\begin{subfigure}{0.5\textwidth}
\centering 
% \vspace{-5em}
\includegraphics[width=\textwidth, right, trim=0 85 0 0, clip]{seeley_connectivity_overlap.jpg}
\caption{Seeley et al., Neuron, 2009}
\end{subfigure}

\end{figure}

\vfill

\vspace{-3em}


\end{frame}


\newcommand{\outFolder}{../overview/modelDiagram}
\newcommand{\lw}{0.5mm}

\newcommand{\yes}{{\LARGE \textcolor{green!50!black}{\checkmark} \par}}
\newcommand{\no}{{\LARGE \textcolor{red}{\xmark} \par}}


\begin{frame}
\frametitle{Method Idea - Combine Unsupervised Learning and Disease Progression Modelling}
% method slide 1

\vspace{-1em}

\begin{columns}[T]
%     \hspace{-2em}
  \begin{column}{.47\textwidth}
  
  \begin{center}
   
  Only Unsupervised Learning (i.e. Clustering)
  
%   \hrulefill
  
  \begin{figure}
  \centering
  \includegraphics[height=3cm]{clust24_drcThFWHM0Initk-meansCl4Pr0Ra1Mrf5_VDPM_MRFPCA.png}
  \end{figure}
  \vspace{-1.5em}
 
  \begin{itemize}
   \item Can identify disconnected atrophy patterns \yes
   \item No biomarker trajectories \no
   \item No disease staging of subjects  \no
  \end{itemize}

 

  
  \end{center}  
  \end{column}
  \hspace{-2em}
  \vrule{}
  \begin{column}{.47\textwidth}
  \begin{center}
    
  Only Disease Progression Modelling
  
%   \hrulefill
  
  \begin{figure}
    \centering
    \includegraphics[height=3cm,trim=120 0 120 0]{Disease_progression_one_sigmoid_confidence.png}
  \end{figure}
  \vspace{-1.5em}

  \begin{itemize}
   \item Cannot identify disconnected atrophy patterns \no
   \item Can estimate biomarker trajectories \yes
   \item Can estimate subjects disease stages \yes
  \end{itemize}

  
  \end{center}
  \end{column}
\end{columns}

\vspace{1.5em}

\begin{itemize}
  \item Estimate trajectories for each vertex on the cortical surface
  \item Vertex measures pathology (e.g. thickness, amyloid) at that location
\end{itemize}


\end{frame}



\begin{frame}[label=current]
\frametitle{DIVE clusters vertices/voxels with similar trajectories of pathology}

\begin{figure}
\centering
\includegraphics[height=5.5cm]{vwdpm_diagram.pdf}
\end{figure}

    
\end{frame}



\begin{frame}
\frametitle{Method Step 1 - Model Disease Progression Scores for Every Subject}

\begin{columns}[T]
    \begin{column}{.7\textwidth}
     %\begin{block}{}
    
%     \textbf{Idea}
%     \begin{itemize}
%       \item Combine two techniques:
%       \begin{itemize}
%       \item unsupervised learning (clustering)
%       \item disease progression modelling
%       \end{itemize}
%       
%       \item Estimate trajectories for each vertex on the cortical surface
%       \item Vertex measures cortical thickness at that location
% 
%     \end{itemize}
    
    
    \vspace{-2em}
   
%     \textbf{Method outline}:
%     \setbeamertemplate{enumerate items}[default]
%      \begin{enumerate}
    Each subject $i$ at visit $j$ has an associated \emph{disease progression score} (DPS) $s_{ij}$:
      $$s_{ij} = \alpha_i t_{ij} + \beta_i$$
      
      where:
      \begin{itemize}
       \item $s_{ij}$ - disease progression score of subject $i$ at timepoint $j$
       \item $t_{ij}$ - age of subject $i$ at timepoint $j$
       \item $\alpha_i $ - progression speed of subject $i$
       \item $\beta_i $ - time shift of subject $i$
      \end{itemize}
            
%      \end{enumerate}
     

    %\end{block}
    \end{column}
    \hspace{-2em}
    \begin{column}{.25\textwidth}
    %\begin{block}{}
    
    \vspace{1em}
    \begin{figure}
    \centering
    \includegraphics[scale=0.15]{disease_axis.png}
    \end{figure}

    %\end{block}
    \end{column}
  \end{columns}


\end{frame}


%%%%%%%%%%%%%%%%%%%%%%%%%%%%%%%%%%%%%%%%%%%%

\begin{frame}
\frametitle{Step 2 - Model Evolution of Pathology at Specific Location in the Brain}
% method slide 2
\begin{columns}[T]
%     \hspace{-2em}
    \begin{column}{.7\textwidth}
     %\begin{block}{}
    

    \setbeamertemplate{enumerate items}[default]
     \begin{itemize}
  
      \item Each biomarker measurement $V_l^{ij}$ follows a sigmoidal curve $f(\cdot\ ;\theta)$ along the disease progression:
      
      $$ V_l^{ij} \approx f(s_{ij};\theta_k) = \frac{a_k}{1+exp(-b_k(s-c_k))} + d_k $$
      
      where
      \begin{itemize}
      \item  $V_l^{ij}$ - biomaker (e.g. thickness, amyloid) at location $l$ for subject $i$, timepoint $j$
       \item $\theta_k = [a_k, b_k, c_k, d_k]$ - parameters of $k$-th sigmoid curve
      \end{itemize}
      
      \vspace{2em}
      
      \item We assume Gaussian noise along the $k$-th trajectory:

      $$p(V_l^{ij} | \alpha_i, \beta_i, \theta_k, \sigma_k) \sim N(f(\alpha_i t_{ij} + \beta_i ; \theta_k), \sigma_k)$$
            
      where:
      \begin{itemize}
       \item $N$ - pdf of the Gaussian distribution
       \item $\sigma_k$ - noise level
       
      \end{itemize}
            
     \end{itemize}
     

    %\end{block}
    \end{column}
    \hspace{-2em}
    \begin{column}{.25\textwidth}
    %\begin{block}{}
    
    \begin{figure}
    \centering
    \includegraphics[scale=0.14, trim=120 0 120 0]{Disease_progression_one_sigmoid_confidence.png}
    \end{figure}

    %\end{block}
    \end{column}
  \end{columns}

\end{frame}



%%%%%%%%%%%%%%%%%%%%%%%%%%%%%%%%%%%%%%%%%%


\begin{frame}
\frametitle{Our Model So Far}
% method slide 3

\begin{columns}[T]
%     \hspace{-4em}
    \begin{column}{.7\textwidth} % TODO remove columns here, not needed anymore
     %\begin{block}{}
   
%     \setbeamertemplate{enumerate items}[default]
     
%     \textbf{Idea:} Group vertices with similar progression dynamics into clusters\\ 
%    \vspace{2em}
%     \textbf{Method outline - continued}:
   \begin{enumerate}      
      
      \item Model disease progression score for one subject $i$ at visit $j$:
      $$s_{ij} = \alpha_i t_{ij} + \beta_i$$
      
      \vspace{1em}
      
      \item Model biomarker trajectory of one vertex (point) on the brain:
      $$p(V_l^{ij} | \alpha_i, \beta_i, \theta_k, \sigma_k) \sim N(f(\alpha_i t_{ij} + \beta_i ; \theta_k), \sigma_k)$$
      
  
     
     \end{enumerate}
     

    %\end{block}
    \end{column}
%     \hspace{-3em}
    \begin{column}{.3\textwidth}

    \vspace{-2em}
    
    \begin{figure}
    \centering
    \includegraphics[height=1.5cm]{disease_axis.png}
    \end{figure}
    
    \begin{figure}
    \centering
    \includegraphics[height=1.5cm, trim=120 0 120 0]{Disease_progression_one_sigmoid_confidence.png}
    \end{figure}
    

    %\end{block}
    \end{column}
  \end{columns}
  
  \vspace{6em}
  
%   \begin{itemize}
% 
%   
%   \end{itemize}


\end{frame}

%%%%%%%%%%%%%%%%%%%%%%%%%%%%%%%%%%%%%%%%%%%%


\begin{frame}
\frametitle{Step 3: Group Vertices with Similar Progression Dynamics into Clusters}
% method slide 3

\begin{columns}[T]
%     \hspace{-4em}
    \begin{column}{.7\textwidth} % TODO remove columns here, not needed anymore
     %\begin{block}{}
   
    \setbeamertemplate{enumerate items}[default]
     
%     \textbf{Idea:} Group vertices with similar progression dynamics into clusters\\ 
%    \vspace{2em}
%     \textbf{Method outline - continued}:
   \begin{itemize}      
      
      \item Define $Z_l$ as the cluster that generated vertex $l$:
      $$ p(V_l^{ij} | \alpha_i, \beta_i, \theta_{Z_l}, \sigma_{Z_l}, Z_l) \sim N(f(\alpha_i t_{ij} + \beta_i ; \theta_{Z_l}), \sigma_{Z_l}) $$
        where
	\begin{itemize}
	\item $Z_l$ - discreete latent variable allocating\\ vertex $l$ to a cluster $k \in [1 \dots K]$
	\end{itemize}
      \vspace{1em}
      \item Extend to all subjects and vertices:
  $$  p(V, Z | \alpha, \beta, \theta, \sigma) = \prod_l^L \prod_{(i,j) \in I} N(V_l^{ij} | f(\alpha_i t_{ij} + \beta_i ; \theta_{Z_l}), \sigma_{Z_l}) $$
  where
  \begin{itemize}
  \item $L$ - the total number of vertices on the cortical surface
  \item $I = {(i,j)}$ - set of available timepoints for each subject $i$ and timepoint $j$   
  \item we assume independence across subjects and voxels in different clusters
  \end{itemize}
     
  \end{itemize}
     

    %\end{block}
    \end{column}
%     \hspace{-3em}
    \begin{column}{.3\textwidth}
    %\begin{block}{}

%         \node  (brain) at (1.3,4.5) {\includegraphics[scale=0.1]{disease_progression_staging.png}};
    
       
    \begin{figure}
    \centering
    \includegraphics[scale=0.28, trim=120 0 120 70]{\outFolder/sigManyBiomkClustering.png}
    \end{figure}
    

    %\end{block}
    \end{column}
  \end{columns}
  
%   \begin{itemize}
% 
%   
%   \end{itemize}


\end{frame}


%%%%%%%%%%%%%%%%%%%%%%%%%%%%%%%%%%%%%%%%%%


\begin{frame}
\frametitle{Our Model So Far}
% method slide 3

\begin{columns}[T]
%     \hspace{-4em}
    \begin{column}{.7\textwidth} % TODO remove columns here, not needed anymore
     %\begin{block}{}
   
%     \setbeamertemplate{enumerate items}[default]
     
%     \textbf{Idea:} Group vertices with similar progression dynamics into clusters\\ 
%    \vspace{2em}
%     \textbf{Method outline - continued}:
   \begin{enumerate}      
      
      \item Model disease progression score for one subject $i$ at visit $j$:
      $$s_{ij} = \alpha_i t_{ij} + \beta_i$$
      
      \vspace{1em}
      
      \item Model biomarker trajectory of one vertex (point) on the brain:
      $$p(V_l^{ij} | \alpha_i, \beta_i, \theta_k, \sigma_k) \sim N(f(\alpha_i t_{ij} + \beta_i ; \theta_k), \sigma_k)$$
      
      \vspace{1em}
      
      \item Extend to all vertices and subjects:
  $$  p(V, Z | \alpha, \beta, \theta, \sigma) = \prod_l^L \prod_{(i,j) \in I} N(V_l^{ij} | f(\alpha_i t_{ij} + \beta_i ; \theta_{Z_l}), \sigma_{Z_l}) $$

      \vspace{1em}
  
     
     \end{enumerate}
     

    %\end{block}
    \end{column}
%     \hspace{-3em}
    \begin{column}{.3\textwidth}

    \vspace{-2em}
    
    \begin{figure}
    \centering
    \includegraphics[height=1.5cm]{disease_axis.png}
    \end{figure}
    
    \begin{figure}
    \centering
    \includegraphics[height=1.5cm, trim=120 0 120 0]{Disease_progression_one_sigmoid_confidence.png}
    \end{figure}
    
    \begin{figure}
    \centering
    \includegraphics[height=1.5cm, trim=120 0 120 70]{\outFolder/sigManyBiomkClustering.png}
    \end{figure}

    %\end{block}
    \end{column}
  \end{columns}
  
  \vspace{6em}
  
%   \begin{itemize}
% 
%   
%   \end{itemize}


\end{frame}


%%%%%%%%%%%%%%%%%%%%%%%%%%%%%%%%%%%%%%%%%%%%


\begin{frame}
\frametitle{Step 4: Marginalise over the Hidden Variables Z (Cluster Assignments)}
% method slide 3

\begin{columns}[T]
%     \hspace{-4em}
    \begin{column}{.7\textwidth} % TODO remove columns here, not needed anymore
     %\begin{block}{}
   
%     \setbeamertemplate{enumerate items}[default]
     
%     \textbf{Idea:} Group vertices with similar progression dynamics into clusters\\ 
%    \vspace{2em}
%     \textbf{Method outline - continued}:
   \begin{enumerate}      
      
      \item Model disease progression score for one subject $i$ at visit $j$:
      $$s_{ij} = \alpha_i t_{ij} + \beta_i$$
      
      \vspace{1em}
      
      \item Model biomarker trajectory of one vertex (point) on the brain:
      $$p(V_l^{ij} | \alpha_i, \beta_i, \theta_k, \sigma_k) \sim N(f(\alpha_i t_{ij} + \beta_i ; \theta_k), \sigma_k)$$
      
      \vspace{1em}
      
      \item Extend to all vertices and subjects:
  $$  p(V, Z | \alpha, \beta, \theta, \sigma) = \prod_l^L \prod_{(i,j) \in I} N(V_l^{ij} | f(\alpha_i t_{ij} + \beta_i ; \theta_{Z_l}), \sigma_{Z_l}) $$

      \vspace{1em}
  
      \item Marginalise over the hidden variables $Z_l$ (cluster assignments):
  \small{$$p(V|\alpha, \beta, \theta, \sigma) = \prod_{l=1}^L \sum_{k=1}^K p(Z_l = k) \prod_{(i,j) \in I} N(V_l^{ij} | f(\alpha_i t_{ij} + \beta_i \ ; \theta_k), \sigma_k)$$}
     
     \end{enumerate}
     

    %\end{block}
    \end{column}
%     \hspace{-3em}
    \begin{column}{.3\textwidth}

    \vspace{-2em}
    
    \begin{figure}
    \centering
    \includegraphics[height=1.5cm]{disease_axis.png}
    \end{figure}
    
    \begin{figure}
    \centering
    \includegraphics[height=1.5cm, trim=120 0 120 0]{Disease_progression_one_sigmoid_confidence.png}
    \end{figure}
    
    \begin{figure}
    \centering
    \includegraphics[height=1.5cm, trim=120 0 120 70]{\outFolder/sigManyBiomkClustering.png}
    \end{figure}

    %\end{block}
    \end{column}
  \end{columns}
  
%   \begin{itemize}
% 
%   
%   \end{itemize}


\end{frame}



\begin{frame}
\frametitle{Step 5: Modelling Spatial Correlation using Markov Random Fields}

\textbf{Motivation}
\begin{itemize}
 \item measurements from neighouring vertices are inherently correlated
 \item can "fill-in holes", eliminate noisy cluster assignments due to noise 
\end{itemize}


% MRF extension
$$ p(V, Z | \alpha, \beta, \theta, \sigma) = \prod_l^L \prod_{(i,j) \in I} N(V_l^{ij} | f(\alpha_i t_{ij} + \beta_i | \theta_{Z_l}), \sigma_{Z_l}) \textcolor{red}{\prod_{l_1 \sim l_2} \Psi (Z_{l_1}, Z_{l_2})}$$

where 
\begin{itemize}
 \item $
 \Psi (Z_{l_1}=k_1, Z_{l_2}=k_2) = 
 \begin{cases}
  exp(\lambda) & \text{if } k_1 = k_2\\
  exp(-\lambda) & \text{otherwise}
 \end{cases}
$
 \item $\lambda$ - MRF parameter
\end{itemize}


\vspace{-1em}

\begin{figure}
\begin{subfigure}{0.3\textwidth}
\centering
 \includegraphics[scale=0.15]{slopeCol_drcThFWHM0Initk-meansCl3Pr0Ra1Mrf5_VWDPMMeanAD.png}
 \caption{Without MRF}
 \end{subfigure}
 \begin{subfigure}{0.3\textwidth}
 \centering
 \includegraphics[scale=0.15]{slopeCol_drcThFWHM0Initk-meansCl3Pr0Ra1Mrf5_VDPM_MRFAD.png}
 \caption{With MRF,  $\alpha = 5$.}
 \end{subfigure}
\end{figure}



\end{frame}





\begin{frame}[label=current]
\frametitle{Model Fitting with Expectation-Maximisation (EM)}

% \newcommand{\mycirc}[2]{\draw (#1,#2) circle (3cm);}

% \newcommand{\outFolder}{.}
% \small{
    \vspace{-4em}
    \begin{itemize}
    \item \textbf{E-step}:
    \begin{itemize}
    \item Estimate vertex assignment to clusters $z_{lk}^{(u)} = \zeta_{lk}(\lambda^{(u)})$:
     
    $$ \lambda^{(u)} = \argmax_{\lambda}\ \sum_{l=1}^L \sum_{k=1}^K \zeta_{lk}(\lambda) \left[  D_{lk} \  + \lambda \sum_{l_2 \in N_l}  \zeta_{l_2 k}(\lambda)\  -\lambda^2 \sum_{l_2 \in N_l} (1- \zeta_{l_2 k}(\lambda))  \right]$$\\
    $$    \zeta_{lk}(\lambda) \approx exp \left( D_{lk} +   \sum_{l_2 \in N_l} log\ \left[ exp(-\lambda^2) + z_{l_2k}^{(u-1)} (exp(\lambda) - exp(-\lambda^2)) \right] \right) $$
    where:
    $$ D_{lk} = -\frac{1}{2}log\ (2 \pi \left(\sigma_k^{(u)}\right)^2) |I| - \frac{1}{2\left(\sigma_k^{(u)}\right)^2} \sum_{i,j \in I} (V_l^{ij} - f(\alpha_i^{(u)} t_{ij} + \beta_i^{(u)} | \theta_k^{(u)}))^2$$
        
    \end{itemize}
    \item \textbf{M-step}:
    \begin{itemize}
     \item Update trajectories:
     
     \begin{equation}
 \label{eq:theta}
 \theta_k = \argmin_{\theta_k} \left[\sum_{l=1}^L z_{lk} \sum_{(i,j) \in I} (V_l^{ij} - f(\alpha_i t_{ij} + \beta_i | \theta_k))^2 \right] - log\ p(\theta_k) 
\end{equation}
     
     \item Update subject progression scores:
     
     \begin{equation}
\label{eq:alpha}
 \alpha_i, \beta_i = \argmin_{\alpha_i, \beta_i}  \left[ \sum_{l=1}^L \sum_{k=1}^K z_{lk} \frac{1}{2\sigma_k^2} \sum_{j \in I_i} (V_l^{ij} - f(\alpha_i t_{ij} + \beta_i | \theta_k))^2\right] - log\ p(\alpha_i, \beta_i)
\end{equation}
     
    \end{itemize}
    \end{itemize}
\vspace{-3em}
    
\end{frame}



\begin{frame}
\frametitle{Methods - Numerical Optimisation and Initialisation}

\textbf{Numerical optimisation} 
\begin{itemize}
\item E- and M-steps have no analytical solution
\item Perform numerical optimisation with Nelder-Mead
\begin{itemize}
  \item robust and fast convergence
\end{itemize}
\item EM still converges with partial E- and M-steps
\end{itemize}
%\end{block}

\vfill

\textbf{Initialisation}

\begin{itemize}
 \item We set $\alpha_i=1$ and $\beta_i=0$, $\forall i$
 \item We initialise $z_{lk} = p(Z_l = k|V_l,\Theta^{old})$ using k-means clustering
 \begin{itemize}
  \item feature vector for vertex $l$: $\left[ V_l^{ij} | (i,j) \in I \right]$ (measurements for all subjects at that location)
 \end{itemize}

 \item Estimate the optimal number of clusters with the Bayesian Information Criterion (BIC)
 \begin{itemize}
  \item Number of parameters: $5K + 2S$
 \end{itemize}

\end{itemize}

\end{frame}


\begin{frame}
\frametitle{DIVE Finds Plausible Atrophy Patterns on Four Datasets}



\newcommand{\scalingFactor}{1.2}
\newcommand{\gradLimLeft}{-1.6}
\newcommand{\gradLimRight}{1.6}

\definecolor{barGreen}{rgb}{0.4,1,0.4}

\begin{itemize}
 \item Similar patterns of tAD atrophy in independent datasets: ADNI and UCL DRC
 \item Distinct patterns of atrophy in different diseases (tAD and PCA) and modalities (MRI vs PET)
\end{itemize}


% FWHM0 avg thickness map MCI & AD
\begin{figure}[h]
  \centering
  \vspace{-1em}

  % do the legend colorbar
  \begin{subfigure}[b]{0.45\textwidth}
   \centering
  \begin{tikzpicture}[scale=\scalingFactor]
    \shade[left color=red,right color=yellow] (\gradLimLeft,2.5) rectangle (-0.8,2.75);
    \shade[left color=yellow,right color=barGreen] (-0.8,2.5) rectangle (0,2.75);
    \shade[left color=barGreen,right color=cyan] (0,2.5) rectangle (0.8,2.75);	
    \shade[left color=cyan,right color=blue] (0.8,2.5) rectangle (\gradLimRight,2.75);   

    \node[inner sep=0] (corr_text) at (\gradLimLeft,3) {severe pathology};
    \node[inner sep=0] (corr_text) at (\gradLimRight,3) {moderate pathology};
  \end{tikzpicture}
  \vspace{0.3em}
  \end{subfigure}
  

  
  \begin{subfigure}[b]{0.25 \textwidth}
   \centering
   \Large{ADNI MRI}
  \includegraphics[width=\textwidth,trim=0 0 0 20,clip]{\voxFld/selected_resfiles/adniThick/atrophyExtent24_adniThInitk-meansCl3Pr1Ra1_VDPM_MRF.png}
  \end{subfigure} 
  ~
  \begin{subfigure}[b]{0.25 \textwidth}
   \centering
   \Large{DRC tAD}
  \includegraphics[width=\textwidth,trim=0 0 0 20,clip]{\voxFld/selected_resfiles/drcAD/atrophyExtent24_drcThInitk-meansCl3Pr1Ra1_VDPM_MRFAD.png}
  \end{subfigure}
  \vspace{0.5em}

  \begin{subfigure}[b]{0.25 \textwidth}
   \centering
   \Large{DRC PCA}
  \includegraphics[width=\textwidth,trim=0 0 0 20,clip]{\voxFld/selected_resfiles/drcPCA/atrophyExtent24_drcThInitk-meansCl5Pr1Ra1_VDPM_MRFPCA.png}
  \end{subfigure}
  ~
  \begin{subfigure}[b]{0.25 \textwidth}
   \centering
   \Large{ADNI PET AV45}
  \includegraphics[width=\textwidth,trim=0 0 0 20,clip]{\voxFld/selected_resfiles/adniPet/atrophyExtent24_adniPetInitk-meansCl18Pr1Ra1_VDPM_MRF.png}
  \end{subfigure}
  
  \small{Marinescu et al., NeuroImage, 2019}\\
  \small{source code: https://github.com/mrazvan22/dive}

\end{figure}


\end{frame}



\begin{frame}
\frametitle{DIVE Estimates the Temporal Evolution of Pathology, Enabling Understanding of Disease Mechanisms}


\newcommand{\scalingFactor}{1.2}
\newcommand{\gradLimLeft}{-1.6}
\newcommand{\gradLimRight}{1.6}

\definecolor{barGreen}{rgb}{0.4,1,0.4}

\newcommand{\speed}{2}

\vfill


% FWHM0 avg thickness map MCI & AD
\begin{figure}[h]
  \centering
  \vspace{-1em}

  % do the legend colorbar
  \begin{subfigure}[b]{0.45\textwidth}
   \centering
  \begin{tikzpicture}[scale=\scalingFactor]
    \shade[left color=red,right color=yellow] (\gradLimLeft,2.5) rectangle (-0.8,2.75);
    \shade[left color=yellow,right color=barGreen] (-0.8,2.5) rectangle (0,2.75);
    \shade[left color=barGreen,right color=cyan] (0,2.5) rectangle (0.8,2.75);	
    \shade[left color=cyan,right color=blue] (0.8,2.5) rectangle (\gradLimRight,2.75);   

    \node[inner sep=0] (corr_text) at (\gradLimLeft,3) {severe pathology};
    \node[inner sep=0] (corr_text) at (\gradLimRight,3) {moderate pathology};
  \end{tikzpicture}
  \vspace{0.3em}
  \end{subfigure}
  

	\begin{animateinline}[autoplay,loop]{\speed}  
   \multiframe{20}{i=10+1}{% loop through pictures
%  \multiframe{1}{i=10+1}{% loop through pictures \multiframe{nrOfPics}{i=initialVal+increment} 
  
  \parbox{\textwidth}{
  \centering
  
  \begin{subfigure}[b]{0.25 \textwidth}
   \centering
   \Large{ADNI MRI}
    \includegraphics[width=\textwidth]{\voxFld/selected_resfiles/adniThick/movie_adniThInitk-meansCl3Pr1Ra1_VDPM_MRFtext_\i}
  \end{subfigure} 
  ~
  \begin{subfigure}[b]{0.25 \textwidth}
   \centering
   \Large{DRC tAD}
  \includegraphics[width=\textwidth]{\voxFld/selected_resfiles/drcAD/movie_drcThInitk-meansCl3Pr1Ra1_VDPM_MRFADtext_\i}
  \end{subfigure}
  \vspace{0.5em}


  \begin{subfigure}[b]{0.25 \textwidth}
   \centering
   \Large{DRC PCA}
    \includegraphics[width=\textwidth]{\voxFld/selected_resfiles/drcPCA/movie_drcThInitk-meansCl5Pr1Ra1_VDPM_MRFPCAtext_\i} 
  \end{subfigure}
  ~
  \begin{subfigure}[b]{0.25 \textwidth}
   \centering
   \Large{ADNI PET AV45}
   \includegraphics[width=\textwidth]{\voxFld/selected_resfiles/adniPet/movie_adniPetInitk-meansCl18Pr1Ra1_VDPM_MRFtext_\i} 
  \end{subfigure}
  
  }  
  }
  \end{animateinline}

  
  \small{Marinescu et al., Neuroimage, 2019}\\
  \small{source code: https://github.com/mrazvan22/dive}

\end{figure}

% \begin{itemize}
%  \item Animations generated using BrainPainter: https://github.com/mrazvan22/brain-coloring
% \end{itemize}

\end{frame}




%%%%%%%%%%%%%%%%%%%%%%%%%%%%%%%%%%%%%%%%%%%%%%%%%%%%%%%%%%%%%

% \newcommand{\ipmiPaperFold}{.}
\newcommand{\outFoldADNICVbrains}{../overview/figures_ipmi_paper/crossvalid/adniThavgFWHM0Initk-meansCl3Pr0Ra1_VWDPMMean}
\newcommand{\scaleFig}{0.16}

\begin{frame}
\frametitle{Validation - Model Robustly Estimates Atrophy Patterns}

\textbf{Method:} Tested the consistency of the spatial clustering in ADNI using 10-fold CV\\
\vspace{1em}

\textbf{Results:} Good agreement in terms of spatial distribution (dice score 0.89)\\

\begin{figure}[h]
    \centering
    
%     \newcounter{classnumber}
    \foreach \n in {0,...,9}{
    \begin{subfigure}[b]{\scaleFig\textwidth}
    \centering
    f=\n \\
    \includegraphics[width=\textwidth]{\outFoldADNICVbrains/blend\n.png}\\
%     \vspace{-1.5em}
    \includegraphics[width=\textwidth,trim=0 0 0 25,clip]{../overview/figures/cogCorr/trajSamplesOneFig_cogCorr_adniThFWHM0Initk-meansCl3Pr0Ra1Mrf5_VWDPMMean_f\n.png}
    \end{subfigure}
    }
    
%     \caption{Cross-validation}
    \label{fig:ADNICVbrains}
\end{figure}
\vspace{-2em}
\begin{center}
\small{Marinescu et al., Neuroimage, 2019}\\
\small{source code: https://github.com/mrazvan22/dive}
\end{center}

\end{frame}

%%%%%%%%%%%%%%%%%%%%%%%%%%%%%%%%%%%%%%%%%%%%%%%%%%%%%%%%%%%%%

\begin{frame}
\frametitle{Estimated Subject Progression Scores are Clinically Relevant}

\vspace{-2em}

\textbf{Hypothesis}: 
\begin{itemize}
 \item Clinical relevance $\rightarrow$ DPS correlates with other markers of disease progression
\end{itemize}

\vfill

\textbf{Method}: Ran our model on ADNI using 10-fold cross-validation

\vfill

\textbf{Results}: Progression scores correlate well with cognitive tests:



\newcommand{\figFont}{\small}
\newcommand{\pValFont}{\tiny}

\begin{figure}[h]
  \begin{subfigure}{0.22\textwidth}
    \centering
    \figFont{CDRSOB}\\ \pValFont{($\rho = 0.41$, $p < 1e-66$)}
    \includegraphics[width=1.1\textwidth]{../overview/figures/stagingCogTestsScatterPlot_adniThFWHM0Initk-meansCl3Pr0Ra1Mrf5_VWDPMMean_ADAS13.png}
  \end{subfigure}
  \begin{subfigure}{0.22\textwidth}
    \centering
    \figFont{ADAS-COG}\\ \pValFont{($\rho = -0.40$, $p < 1e-62$)}
    \includegraphics[width=1.1\textwidth]{../overview/figures/stagingCogTestsScatterPlot_adniThFWHM0Initk-meansCl3Pr0Ra1Mrf5_VWDPMMean_MMSE.png}
  \end{subfigure}
    \begin{subfigure}{0.22\textwidth}
    \centering
    \figFont{MMSE}\\ \pValFont{($\rho = -0.39$, $p < 1e-58$)}
    \includegraphics[width=1.1\textwidth]{../overview/figures/stagingCogTestsScatterPlot_adniThFWHM0Initk-meansCl3Pr0Ra1Mrf5_VWDPMMean_RAVLT.png}
  \end{subfigure}
    \begin{subfigure}{0.22\textwidth}
    \centering
    \figFont{RAVLT}\\ \pValFont{($\rho = 0.39$, $p < 1e-58$)}
    \includegraphics[width=1.1\textwidth]{../overview/figures/stagingCogTestsScatterPlot_adniThFWHM0Initk-meansCl3Pr0Ra1Mrf5_VWDPMMean_CDRSOB.png}
  \end{subfigure}
\end{figure}

\vspace{-2em}


\end{frame}



\begin{frame}[label=current]
\frametitle{Future research directions}

\textbf{Modelling}:
\begin{itemize}
\item Incorporate biological mechanisms (Raj et al., 2012,  Georgiadis et al., 2018)
\item Incorporate other sources of data: e.g. genetics (Sclesi et al, Brain, 2018)
\item Account for heterogeneity (e.g. Young et al., Nature Comm., 2018)
\end{itemize}

\vfill

\textbf{Simulations}: 
\begin{itemize}
\item Use disease progression models to simulate cohorts (Koval et al., arXiv, 2019)
\end{itemize}

\vfill

\textbf{Applications}:
\begin{itemize}
\item Other NDs: Multiple Sclerosis, Huntington's, Parkinson's
\item Other pathologies: e.g. tumours, lesions
\end{itemize}

\end{frame}


% \begin{frame}[label=current]
% \frametitle{BrainPainter features}
% 
% Brain types:
% \begin{figure}
% \centering
% 
% pial
% 
% 
% inflated
% 
% \end{figure}
% 
% 
% \end{frame}


\begin{frame}[label=current]
\frametitle{Acknowledgements}

% \includegraphics[width=\textwidth,trim=0 0 0 150, clip]{cmic_away_day}
 
 \begin{small}
\begin{columns}[T]
%     \hspace{-4em}
    \begin{column}{.3\textwidth}
    \textbf{Collaborators}
    \begin{enumerate}
    \item Leon Aksman
    \item Maura Bellio
     \item Arman Eshaghi
     \item Nicholas Firth
     \item Sara Garbarino
     \item Marco Lorenzi
     \item Kyriaki Mengoudi
     \item Neil Oxtoby
     \item Alexandra Young
    \end{enumerate}
%     \includegraphics[scale=1]{pondLogo.png}  
    \end{column}
    
    
%     \begin{column}{.6\textwidth}
%     \begin{center}
%     \textbf{Project Supervisors}
%     \end{center}
%     \vspace{-2em}
%       \begin{figure}
%       \begin{subfigure}{0.3\textwidth}
%       \centering
%       Daniel Alexander
%       \includegraphics[height=2cm]{Danny-Alexander.jpeg}  
%       \end{subfigure}
% 	\begin{subfigure}{0.3\textwidth}
% 	\centering
%       Sebastian Crutch
%       \includegraphics[height=2cm]{Seb_Crutch_photo.JPG}  
%       \end{subfigure}
%   \begin{subfigure}{0.3\textwidth}
% 	\centering
%       Polina Golland
%       \includegraphics[height=2cm, trim=0 00 0 0,clip]{polina}  
%       \end{subfigure}
%       
%     \begin{center}
%     \textbf{Funders}
%     \end{center}
%     %\vspace{-2em}
%       \begin{subfigure}{0.3\textwidth}
%       \centering
%       \includegraphics[height=1cm]{epsrc_logo}  
%       \end{subfigure}
% 	\begin{subfigure}{0.3\textwidth}
% 	\centering
%       \includegraphics[height=1cm]{CDTlogo}  
%       \end{subfigure}
%      
%       
%       \vspace{2em}
% 
% \end{figure}
%     
% \end{column}
\end{columns}
 
\end{small}


\end{frame}




\end{document}



