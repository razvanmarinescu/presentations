\documentclass[8pt,xcolor=table]{beamer}

\usepackage{graphicx}
\usepackage{caption}
\usepackage{subcaption}
\usepackage{transparent}
\usepackage{epstopdf} %converting to PDF
\usepackage{multicol} 
\usepackage{animate}[2017/05/18]

% \usepackage{pdfx}
 
% \usepackage[utf8]{inputenc}
% \usepackage[T1]{fontenc}
\usepackage[table]{xcolor}    % loads also »colortbl« 
%  \usepackage{enumitem}
% \usepackage{ucltemplate}
\usepackage{color}

\usepackage{comment}

\usepackage{tabularx} % make width of table columns evenly distributed (see http://tex.stackexchange.com/questions/60601/evenly-distributing-column-widths)
% \newcolumntype{Y}{>{\centering\arraybackslash}X}

% make entire row bold or italic in table
\newcommand\setrow[1]{\gdef\rowmac{#1}#1\ignorespaces}
\newcommand\clearrow{\global\let\rowmac\relax}
\clearrow


\usepackage{amssymb}% http://ctan.org/pkg/amssymb
\usepackage{pifont}% http://ctan.org/pkg/pifont
\newcommand{\cmark}{\ding{51}}%
\newcommand{\xmark}{\ding{55}}%


%\usepackage{pgfgantt} % for grantt charts
\usepackage{rotating}
\usepackage[graphicx]{realboxes}
\usepackage[export]{adjustbox}
\usepackage{array}

\usepackage{rotating}
% \usepackage{tabularx, booktabs} % make width of table columns evenly distributed (see http://tex.stackexchange.com/questions/60601/evenly-distributing-column-widths)
% \newcolumntype{Y}{>{\centering\arraybackslash}X}

\DeclareMathOperator*{\argmin}{arg\,min}
\DeclareMathOperator*{\argmax}{arg\,max}

\usepackage{tikz}
\usetikzlibrary{arrows,positioning, shapes.symbols,shapes.callouts,patterns,shapes,chains,calc,backgrounds,fadings}

% \definecolor{parCol}{rgb}{0.1, 0.1, 1}
% \definecolor{stCol}{rgb}{0.1, 0.6, 0.1}
% \definecolor{bothCol}{rgb}{0, 0.5, 0.5}

\definecolor{parCol}{rgb}{0, 0, 0}
\definecolor{stCol}{rgb}{0, 0, 0}
\definecolor{bothCol}{rgb}{0, 0, 0}
\definecolor{blue3}{HTML}{86B7FC} % med blue
\definecolor{blue1}{HTML}{B5F1FF} % light blue
\definecolor{blue2}{HTML}{E0F9FF} % very light blue

\newcolumntype{C}[1]{>{\centering\let\newline\\\arraybackslash\hspace{0pt}}m{#1}}

\setlength{\tabcolsep}{0.2em}

 
 %% OVERVIEW OF WORK SO FAR %%
 
%Information to be included in the title page:
\title{Medical Image Generation and Analysis using Bayesian Generative Models}
\author[Raz]{
R\u{a}zvan V. Marinescu\vspace{1em}}

\institute{\small{Massachusetts Institute of Technology}

% \vspace{0em}
% \small{Centre for Medical Image Computing, University College London, UK}
}

\date{}

% logo of my university
\titlegraphic{
   \begin{figure}
%    \begin{subfigure}{0.32\textwidth}
%    \hspace{2em}
%    \includegraphics[height=1.0cm]{ucl_logo}
%    \end{subfigure}
   \begin{subfigure}{0.32\textwidth}
   \centering
   \includegraphics[height=1.0cm]{MIT_logo.png} 
   \end{subfigure}
%    \begin{subfigure}{0.32\textwidth}
%    \centering
%    \includegraphics[height=1.0cm]{pondLogo.png} 
%    \end{subfigure}
   \end{figure}
   
%    \tiny{Slides available online: https://people.csail.mit.edu/razvan/talk/martinos2019/pres.pdf}
}

\setbeamercolor{frametitle}{fg=black}
\setbeamercolor{author in head/foot}{fg=black, bg=white} 
\setbeamercolor{institute in head/foot}{fg=black, bg=white} 
\setbeamercolor{title in head/foot}{fg=black, bg=white}
\setbeamercolor{date in head/foot}{fg=black, bg=white}

\setbeamersize{text margin left=10pt,text margin right=10pt}
% \setbeamertemplate{frametitle}{
%     \vspace{0.9em}
%     \insertframetitle
% %     \vspace{-3em}
% }
\setbeamertemplate{frametitle}{%
    \vspace{0.5em}
    \usebeamerfont{frametitle}\insertframetitle%
    \vphantom{g}% To avoid fluctuations per frame
    %\hrule% Uncomment to see desired effect, without a full-width hrule
    \par% <-- added
    \hspace*{-\dimexpr0.5\paperwidth-0.5\textwidth}% <-- calculation of left margin width
    \rule[0.5\baselineskip]{\paperwidth}{0.4pt}%
}

\setbeamertemplate{footline}
{
  \vspace{-3em}
  \leavevmode%
   \rule{\paperwidth}{0.3pt}
  \hbox{%
  \begin{beamercolorbox}[wd=.2\paperwidth,ht=2.25ex,dp=1ex,center]{author in head/foot}%
    \usebeamerfont{author in head/foot}Razvan V. Marinescu
  \end{beamercolorbox}%
  \begin{beamercolorbox}[wd=.2\paperwidth,ht=2.25ex,dp=1ex,center]{institute in head/foot}%
    \usebeamerfont{institute in head/foot}razvan@csail.mit.edu
  \end{beamercolorbox}%
  \begin{beamercolorbox}[wd=.3\paperwidth,ht=2.25ex,dp=1ex,center]{institute in head/foot}%
    \usebeamerfont{institute in head/foot}https://people.csail.mit.edu/razvan/
  \end{beamercolorbox}%
  \begin{beamercolorbox}[wd=.2\paperwidth,ht=2.25ex,dp=1ex,center]{title in head/foot}%
    \usebeamerfont{title in head/foot}\insertsection
  \end{beamercolorbox}%
  \begin{beamercolorbox}[wd=.10\paperwidth,ht=2.25ex,dp=1ex,right]{date in head/foot}%
    \usebeamerfont{date in head/foot}\insertshortdate{}\hspace*{2em}
    \insertframenumber{} / \inserttotalframenumber\hspace*{2ex}
  \end{beamercolorbox}}%
  \vskip0pt%
}

% \usepackage{beamerthemesplit}

\newcommand{\backupbegin}{
   \newcounter{finalframe}
   \setcounter{finalframe}{\value{framenumber}}
}
\newcommand{\backupend}{
   \setcounter{framenumber}{\value{finalframe}}
}


\makeatletter
\long\def\beamer@author[#1]#2{%
  \def\and{\tabularnewline}
  \def\insertauthor{\def\inst{\beamer@insttitle}\def\and{\tabularnewline}%
  \begin{tabular}{rl}#2\end{tabular}}%
  \def\beamer@shortauthor{#1}%
  \ifbeamer@autopdfinfo%
    \def\beamer@andstripped{}%
    \beamer@stripands#1 \and\relax
    {\let\inst=\@gobble\let\thanks=\@gobble\def\and{, }\hypersetup{pdfauthor={\beamer@andstripped}}}
  \fi%
}
\makeatother
\beamertemplatenavigationsymbolsempty
\setbeamertemplate{caption}[numbered]
\setbeamercolor{caption name}{fg=black}
\setbeamercolor{itemize item}{fg=black}
\setbeamercolor{itemize subitem}{fg=black}
\setbeamercolor{enumerate item}{fg=black}
\setbeamercolor{enumerate subitem}{fg=black}
\setbeamertemplate{enumerate item}[default]
\setbeamertemplate{enumerate subitem}[default]

\makeatletter
\let\@@magyar@captionfix\relax
\makeatother
\begin{document}
 
\section{Introduction}

\frame{\titlepage}
 
\setbeamerfont{frametitle}{size=\large}

\newcommand{\upgradeReportLoc}{../../upgrade_report}
\newcommand{\epsrcPresLoc}{\upgradeReportLoc/epsrcPres}
\newcommand{\jointModellingDiseaseLoc}{../../jointModellingDisease}
\newcommand{\pcaLongPaperLoc}{../../PCA_long_paper}
\newcommand{\voxFld}{../../voxelwiseDPM}
\newcommand{\tadpoleFld}{../../tadpole}
\newcommand{\diffEqModelFld}{../../diffEqModel}



\newcommand*{\pcaLongFigs}{\pcaLongPaperLoc/figures}


% \includeonlyframes{1-20}
%\includeonlyframes{current}



\newcommand{\ovHeight}{2cm}
\newcommand{\vo}{\vspace{1em}}
\newcommand{\vt}{\vspace{2em}}
\newcommand{\vth}{\vspace{3em}}


% % TODO continue with overview, move into commands
\newcommand{\ovEBM}{
\begin{subfigure}{0.47\textwidth}
\centering
1. Modelled progression of PCA and tAD\\
(using existing methods)
\includegraphics[height=\ovHeight]{ebm_thumb.png}
\end{subfigure}
}

\newcommand{\ovVWDPM}{
\begin{subfigure}{0.47\textwidth}
\centering
% \vspace{2.8em}
2. Developed Novel Spatio-temporal Model \\ (DIVE)\\
\includegraphics[height=\ovHeight]{\upgradeReportLoc/images/vwdpm/blend14_adniThavgFWHM0InithistCl3Pr0Ra1_VWDPMStd.png}
\end{subfigure}
}


\newcommand{\ovDKT}{
\begin{subfigure}{0.47\textwidth}
\centering
\vspace{2em}
3. Developed Novel Transfer Learning \\ method (DKT) \\
\vspace{0.5em}
\includegraphics[height=2.2cm]{\jointModellingDiseaseLoc/paper/figures/disease_knowledge_transfer.pdf}
\end{subfigure}
}


\newcommand{\ovTadpole}{
\begin{subfigure}{0.47\textwidth}
\centering
\vspace{-2em}
4. Organised TADPOLE Competition\\
\vspace{1em}
\includegraphics[height=1.2cm,valign=t]{\upgradeReportLoc/epsrcPres/tadpole} 
\end{subfigure}
}

\newcommand{\ovPainter}{
\begin{subfigure}{\textwidth}
\centering
\vspace{0.5em}
5. Created BrainPainter software\\
\includegraphics[height=1.5cm]{cortical-front_1}\includegraphics[height=1.5cm]{cortical-back_1}\includegraphics[height=1.5cm]{subcortical_1}
\end{subfigure}
}


\definecolor{light-gray}{gray}{0.6}


\newcommand{\inc}[1]{\includegraphics[width=\columnwidth, trim=4 4 4 4, clip]{#1}}
\newcommand{\incw}[2]{\includegraphics[width=#2\columnwidth, trim=4 4 4 4, clip]{#1}}

\begin{frame}{Machine Learning algorithms have achieved impressive milestones}

\begin{columns}[t]
\begin{column}{0.5\textwidth}
\centering
\begin{figure}
\vspace{-2em}

 Object detection (YOLO)
 \incw{yolo}{0.8}
 
 \vt
 
  Image Generation (StyleGAN2)
 \incw{stylegan_small}{0.8}
\end{figure}

\end{column}
\begin{column}{0.5\textwidth}
\centering
Text-to-Image Generation (DALL-E)
\inc{dalle2}
prompt: ``an armchair in the shape of an avocado''

\vspace{1.5em}

Text generation (GPT-3)
\inc{gpt3}
\end{column}

\end{columns}

\vo 

\begin{itemize}
 \item Largely driven by increases in data and compute
\end{itemize}

 
 
\end{frame}


\begin{frame}{However, such milestones have not been translated to medical applications}

\begin{columns}
\begin{column}{0.5\textwidth}
\centering

\begin{itemize}
\item Prediction of clinical variables not always working: 
\begin{itemize}
\item No algorithm, out of 33, could predict ADAS scores in Alzheimer's disease (TADPOLE Challenge, Marinescu et al., 2020)
\end{itemize}
\vspace{2em}

\item Generated images are crude, not high-resolution, mostly 2D

\vspace{2em}

\item Segmentation of pathologies and organs still inaccurate:
\begin{itemize}
\item SOTA placenta segmentation is \textcolor{red}{semi}-automatic
\end{itemize}


\end{itemize}


\end{column}
\begin{column}{0.5\textwidth}
\centering
Brain MRI generation (Han, 2018)
\begin{figure}
\inc{brain-gen}

\vo

%Pneumonia segmentation (Wang, 2020)
%\incw{medseg}{0.7}

Placenta segmentation (Wang, 2020)
\incw{semi-automatic}{0.7}


\end{figure}
\end{column}
\end{columns}

 
 
\end{frame}


\begin{frame}{Why are Machine Learning models not working in medical imaging?}

\vspace{-4em}


\begin{columns}[t]
\begin{column}{0.5\textwidth}
\centering

\begin{itemize}
\item Lack of good labels
\begin{itemize}
 \item Categorical instead of continuous
 \incw{severity}{0.7}
\end{itemize}

 \vspace{2em}
 
 \item Noisy, lack of ground truth
\begin{itemize}
    \item Alzheimer's clinical diagnois: Sensitivity 70\%--87\%; specificity 44\%--70\%  
    \item Gold standard = biopsy
\end{itemize}
\vo 

 \incw{goldstandard}{0.9}


\end{itemize}


\end{column}
\begin{column}{0.5\textwidth}
\centering

\begin{itemize}
\item Lack of good input data

% \begin{itemize}
% \item Stroke scans with limited contrast
% \end{itemize}
Stroke scans with limited contrast
\incw{strokescan}{0.7}

\vspace{2em}

\item Small datasets, inability to scale 

\begin{itemize}
\item most have  $<100$ scans (Maier-Hein, 2018)
\end{itemize}
\incw{imgcount}{0.7}


\end{itemize}


\end{column}
\end{columns}

 
 
\end{frame}


\begin{frame}{What can we do?}


\vspace{-3em}
\begin{columns}[t]
\begin{column}{0.5\textwidth}
\centering

\begin{itemize}
\item Lack of good labels
\begin{itemize}
 \item Infer continuous disease staging
 
%  \incw{severity}{0.7}
\end{itemize}

 \vspace{2em}
 
 \item Noisy, lack of ground truth
\begin{itemize}
   \item Infer true labels from noisy ones  
\end{itemize}

% \incw{goldstandard}{0.7}
\end{itemize}

\vspace{2em}

% \vspace{-4em}

\end{column}
\begin{column}{0.5\textwidth}
\centering

% \vspace{-1.5em}

\begin{itemize}
\item Lack of good input data

\begin{itemize}
  \item Reconstruct better images (compressed sensing)
\end{itemize}
% \incw{strokescan}{0.7}

\vspace{0.5em}

\item Small datasets, inability to scale 
\begin{itemize}
\item Acquire more data
\item Design algorithms for small datasets
\begin{itemize}
 \item Transfer learning
 \item Few-shot or Zero-shot learning
\end{itemize}

\end{itemize}
% \incw{imgcount}{0.7}
\end{itemize}

\vspace{2em}




\end{column}
\end{columns}


\begin{columns}[t]
\begin{column}{0.5\textwidth}
\centering

Solution: Time-series model with\\ latent disease stage\\
= Disease Progression Model\\
%\includegraphics[height=\ovHeight]{\upgradeReportLoc/images/vwdpm/blend14_adniThavgFWHM0InithistCl3Pr0Ra1_VWDPMStd.png}
\includegraphics[height=3cm]{dpm_small}

\end{column}
\begin{column}{0.5\textwidth}
\centering

Solution: Image Reconstruction\\ using Deep Generative Models\\
\includegraphics[height=2cm, trim=6 6 6 6,clip]{brgm_diagram_small}

\end{column}
\end{columns}







\end{frame}


\begin{frame}{Outline}

\begin{itemize}
 \item 
\end{itemize}
 


\end{frame}


% \begin{frame}
% \frametitle{Aim: Estimate the progression of Alzheimer's disease}
% 
% Estimate the progression of Alzheimer's disease 
% 
% 
% 
% \end{frame}

\section{Disease Progression Modelling}

\begin{frame}
\frametitle{Alzheimer's Disease is a Devastating Disease}

\vspace{-1em}
\begin{itemize}
 \item 46 million people affected worldwide
 
  \begin{figure}
 \centering
%   \includegraphics[height=3cm]{adPrelavence}
  \includegraphics[height=4cm]{adPrevalanceIncreasing}
 \end{figure}
 
 \onslide<2-> \item No treatments available that stop or slow down cognitive decline
 \onslide<2-> \item Q: Why did clinical trials fail? A: Treatments were not administered early enough 
 \vspace{1em}
 \onslide<3-> \item Q: How can we then identify subjects \textbf{early} in order to administer treatments? 
 \onslide<3-> \item A: Disease progression model ...
 

\end{itemize}

\vspace{-1em}

\end{frame}

\newcommand{\sz}{1}

\begin{frame}{Building a Disease Progression Model}

\begin{columns}
\begin{column}{0.80\textwidth}
 
% \vspace{-2em}
\begin{center}
\begin{overprint}
\onslide<1> \incw{dps_series9}{\sz}
\onslide<2> \incw{dps_series8}{\sz}
\onslide<3> \incw{dps_series7}{\sz}
\onslide<4> \incw{dps_series6}{\sz}
\onslide<5> \incw{dps_series5}{\sz}
\onslide<6> \incw{dps_series4}{\sz}
\onslide<7> \incw{dps_series3}{\sz}
\onslide<8> \incw{dps_series2}{\sz}
\onslide<9> \incw{dps_series1}{\sz}
\onslide<10-> \incw{dps_series0}{\sz}
\end{overprint}
\end{center}

\begin{itemize}
\onslide<11-> \item Can now build population model
\onslide<12-> \item Early diagnosis
\end{itemize}
\end{column}
\begin{column}[t]{0.2\textwidth}
\vspace{-8em}

\onslide<13-> Previous models:
\onslide<13-> \begin{itemize}
 \item Jedynak, 2012
 \item Fontejin, 2012 
 \item Donohue, 2014
 \item Schiratti, 2017
 \item Lorenzi, 2019
\end{itemize}

\vspace{2em}

\onslide<14-> Limitation: require brain segmentation a-priori

\end{column}


\end{columns}



\end{frame}



% \begin{frame}
% \frametitle{Previous Disease Progression Models required Brain Segmentation ``a-priori''}
% 
% \newcommand{\mnpHeight}{3cm}
% 
% \vspace{-3em}
% % \textbf{Background}:
% \begin{itemize}
% %   \item Modelling the progression of Alzheimer's disease can potentially help drug development
%  \item Several data-driven disease progression models have been recently formulated:
%  
% %   \vfill 
%  
%  \hspace{-2em}
%  \begin{small}
%  \begin{figure}[h]
%  \centering
% %    \begin{minipage}[t][\mnpHeight][t]{0.49\linewidth}
% %   \centering
% %    \textbf{Event-Based Model}\\ 
% %    \begin{subfigure}{0.57\textwidth}
% %    \vspace{-6em}
% %    \includegraphics[width=\textwidth,trim=0 0 450 0,clip]{young_progression2}
% %    
% % %     \includegraphics[width=\textwidth,trim=450 0 0 0,clip]{young_progression2} 
% %    \end{subfigure}
% % %     \vspace{1em}
% %    \includegraphics[width=0.4\textwidth]{young_positional_variance}
% %   \end{minipage}
%   \begin{minipage}[t][\mnpHeight][t]{0.49\linewidth}
%    \centering
%    \textbf{Disease Progression Score}\\ 
%    \includegraphics[width=0.8\textwidth,trim=0 80 0 0, clip]{dps_diagram}
%   \end{minipage}
%     \begin{minipage}[t][\mnpHeight][t]{0.49\linewidth}
%    \centering
%    \textbf{Self-Modelling Regression}\\ 
%    \includegraphics[width=0.8\textwidth]{semor_diagram_cropped}
%   \end{minipage}
% 
% 
% %   \begin{minipage}[t][\mnpHeight][t]{0.49\linewidth}
% %    \centering
% %    \textbf{Manifold-based model}\\ 
% %    \includegraphics[width=1\textwidth,trim=0 270 0 0, clip]{schiratti}
% %    
% %    \vspace{2em}
% %    
% %    
% %   \end{minipage}
% 
% 
%  \end{figure}
%  \end{small}
%  
% %   \vspace{-5em}
%  
%  \item Yet, all these models assumed extracted features from contiguous regions (i.e. atlas-based)
% %  \item We aim to avoid imposing spatial correlation
%  
% \end{itemize}
% 
% \end{frame}




\begin{frame}
\frametitle{Aim: Build a disease progression model for voxelwise data}


% \begin{itemize}
\textbf{Aim}: Move from segmentation-based analysis to voxelwise
% \end{itemize}

\begin{figure}
 \centering
  \begin{tikzpicture}[scale=1]
     \node (roi) at (0,0) {\includegraphics[scale=0.10]{clust24_drcThFWHM0InitfsurfCl4Pr0Ra1Mrf5_VWDPMStaticPCA.png}};
     \node (vw) at (4,0) {\includegraphics[scale=0.10]{clust24_drcThFWHM0Initk-meansCl4Pr0Ra1Mrf5_VDPM_MRFPCA.png}};
     \draw[line width=1.5,->] (roi) -> (vw);
  \end{tikzpicture}
\end{figure}



\begin{figure}
\begin{subfigure}{0.48\textwidth}
\textbf{Why}:
\begin{enumerate}
\item Atrophy correlates with functional networks, which are not spatially connected (Seeley et al., Neuron, 2009)
\vspace{2em}
\item Better biomarker prediction and disease staging
\end{enumerate}
\end{subfigure}
% \hspace{1em}
\begin{subfigure}{0.5\textwidth}
\centering 
% \vspace{-5em}
\includegraphics[width=\textwidth, right, trim=0 85 0 0, clip]{seeley_connectivity_overlap.jpg}
Seeley et al., Neuron, 2009
\end{subfigure}

\end{figure}

\vfill

\vspace{-3em}


\end{frame}







% \begin{frame}
% \frametitle{Background - Disease Progression Modelling}
% 
% \newcommand{\mnpHeight}{3cm}
% 
% \vspace{-3em}
% % \textbf{Background}:
% \textbf{Voxelwise disease progression model} (Bilgel et al., IPMI, 2015)
% \begin{itemize}
%   \item Built on PET data measuring amyloid load at each voxel
%   \item Estimates a unique trajectory for each voxel
%   \item However, it uses a spatial correlation function
% 
%   \vspace{2em}
%   \includegraphics[width=0.85\textwidth]{bilgel_neuroimage}
%   \vspace{2em}
% 
% 
%   
%   \end{itemize}
% 
% 
% 
% \end{frame}






%\begin{frame}
%\frametitle{Method Idea - Combine Unsupervised Learning and Disease Progression Modelling}
%% method slide 1
%
%\vspace{-1em}
%
%\begin{columns}[T]
%%     \hspace{-2em}
%  \begin{column}{.47\textwidth}
%  
%  \begin{center}
%   
%  Only Unsupervised Learning (i.e. Clustering)
%  
%%   \hrulefill
%  
%  \begin{figure}
%  \centering
%  \includegraphics[height=3cm]{clust24_drcThFWHM0Initk-meansCl4Pr0Ra1Mrf5_VDPM_MRFPCA.png}
%  \end{figure}
%  \vspace{-1.5em}
% 
%  \begin{itemize}
%   \item Can identify disconnected atrophy patterns \yes
%   \item No biomarker trajectories \no
%   \item No disease staging of subjects  \no
%  \end{itemize}
%
% 
%
%  
%  \end{center}  
%  \end{column}
%  \hspace{-2em}
%  \vrule{}
%  \begin{column}{.47\textwidth}
%  \begin{center}
%    
%  Only Disease Progression Modelling
%  
%%   \hrulefill
%  
%  \begin{figure}
%    \centering
%    \includegraphics[height=3cm,trim=120 0 120 0]{Disease_progression_one_sigmoid_confidence.png}
%  \end{figure}
%  \vspace{-1.5em}
%
%  \begin{itemize}
%   \item Cannot identify disconnected atrophy patterns \no
%   \item Can estimate biomarker trajectories \yes
%   \item Can estimate subjects disease stages \yes
%  \end{itemize}
%
%  
%  \end{center}
%  \end{column}
%\end{columns}
%
%\vspace{1.5em}
%
%\begin{itemize}
%  \item Estimate trajectories for each vertex on the cortical surface
%  \item Vertex measures pathology (e.g. thickness, amyloid) at that location
%\end{itemize}
%
%
%\end{frame}



\begin{frame}[label=current]
\frametitle{DIVE clusters vertices/voxels with similar trajectories of pathology}

\begin{figure}
\centering
\includegraphics[height=5.5cm]{vwdpm_diagram.pdf}
\end{figure}

    
\end{frame}


\begin{frame}
\frametitle{Building the model using a generative Bayesian framework}
% method slide 3

\begin{columns}[T]
%     \hspace{-4em}
    \begin{column}{.7\textwidth} % TODO remove columns here, not needed anymore
     %\begin{block}{}
   
%     \setbeamertemplate{enumerate items}[default]
     
%     \textbf{Idea:} Group vertices with similar progression dynamics into clusters\\ 
%    \vspace{2em}
%     \textbf{Method outline - continued}:
   \begin{enumerate}      
      
      \onslide<1-> \item Model disease progression score for one subject $i$ at visit $j$:
      $$s_{ij} = \alpha_i t_{ij} + \beta_i$$
      
      \vspace{1em}
      
      \onslide<2-> \item Model biomarker trajectory of one vertex (point) $l$ on the brain:
      \onslide<2-> $$p(V_l^{ij} | \alpha_i, \beta_i, \theta_k, \sigma_k) \sim N(f(\alpha_i t_{ij} + \beta_i ; \theta_k), \sigma_k)$$
      
      \vspace{1em}
      
      \onslide<3-> \item Extend to all vertices and subjects:
      \onslide<3-> $$  p(V, Z | \alpha, \beta, \theta, \sigma) = \prod_l^L \prod_{(i,j) \in I} N(V_l^{ij} | f(\alpha_i t_{ij} + \beta_i ; \theta_{Z_l}), \sigma_{Z_l}) $$

      \vspace{1em}
  
      \onslide<4-> \item Marginalise over the hidden variables $Z_l$ (cluster assignments):
      \onslide<4-> \small{$$p(V|\alpha, \beta, \theta, \sigma) = \prod_{l=1}^L \sum_{k=1}^K p(Z_l = k) \prod_{(i,j) \in I} N(V_l^{ij} | f(\alpha_i t_{ij} + \beta_i \ ; \theta_k), \sigma_k)$$}
     
     \end{enumerate}
     

    %\end{block}
    \end{column}
%     \hspace{-3em}
    \begin{column}{.3\textwidth}

    \vspace{-2em}
    
    \onslide<1-> \begin{figure}
    \centering
    \includegraphics[height=2cm]{disease_axis.png}
    \end{figure}
    
    \onslide<2-> \begin{figure}
    \centering
    \includegraphics[height=2cm, trim=120 0 120 0]{Disease_progression_one_sigmoid_confidence.png}
    \end{figure}
    
    \onslide<3-> \begin{figure}
    \centering
    \includegraphics[height=2cm, trim=120 0 120 70]{\outFolder/sigManyBiomkClustering.png}
    \end{figure}

    %\end{block}
    \end{column}
  \end{columns}
  
%   \begin{itemize}
% 
%   
%   \end{itemize}


\end{frame}






% \begin{frame}
% \frametitle{Methods - Numerical Optimisation and Initialisation}
% 
% \textbf{Numerical optimisation} 
% \begin{itemize}
% \item E- and M-steps have no analytical solution
% \item Perform numerical optimisation with Nelder-Mead
% \begin{itemize}
%   \item robust and fast convergence
% \end{itemize}
% \item EM still converges with partial E- and M-steps
% \end{itemize}
% %\end{block}
% 
% \vfill
% 
% \textbf{Initialisation}
% 
% \begin{itemize}
%  \item We set $\alpha_i=1$ and $\beta_i=0$, $\forall i$
%  \item We initialise $z_{lk} = p(Z_l = k|V_l,\Theta^{old})$ using k-means clustering
%  \begin{itemize}
%   \item feature vector for vertex $l$: $\left[ V_l^{ij} | (i,j) \in I \right]$ (measurements for all subjects at that location)
%  \end{itemize}
% 
%  \item Estimate the optimal number of clusters with the Bayesian Information Criterion (BIC)
%  \begin{itemize}
%   \item Number of parameters: $5K + 2S$
%  \end{itemize}
% 
% \end{itemize}
% 
% \end{frame}


\begin{frame}
\frametitle{DIVE Finds Plausible Atrophy Patterns on Four Datasets}



\newcommand{\scalingFactor}{1.1}
\newcommand{\gradLimLeft}{-1.6}
\newcommand{\gradLimRight}{1.6}

\definecolor{barGreen}{rgb}{0.4,1,0.4}

\begin{itemize}
 \item Similar patterns of tAD atrophy in independent datasets: ADNI and UCL DRC
 \item Distinct patterns of atrophy in different diseases (tAD and PCA) and modalities (MRI vs PET)
\end{itemize}


% FWHM0 avg thickness map MCI & AD
\begin{figure}[h]
  \centering
  \vspace{-1em}

  % do the legend colorbar
  \begin{subfigure}[b]{0.45\textwidth}
   \centering
  \begin{tikzpicture}[scale=\scalingFactor]
    \shade[left color=red,right color=yellow] (\gradLimLeft,2.5) rectangle (-0.8,2.75);
    \shade[left color=yellow,right color=barGreen] (-0.8,2.5) rectangle (0,2.75);
    \shade[left color=barGreen,right color=cyan] (0,2.5) rectangle (0.8,2.75);	
    \shade[left color=cyan,right color=blue] (0.8,2.5) rectangle (\gradLimRight,2.75);   

    \node[inner sep=0] (corr_text) at (\gradLimLeft,3) {severe pathology};
    \node[inner sep=0] (corr_text) at (\gradLimRight,3) {moderate pathology};
  \end{tikzpicture}
  \vspace{0.3em}
  \end{subfigure}
  

  
  \begin{subfigure}[b]{0.2 \textwidth}
   \centering
   \Large{ADNI MRI}
  \includegraphics[width=\textwidth,trim=0 0 0 20,clip]{\voxFld/selected_resfiles/adniThick/atrophyExtent24_adniThInitk-meansCl3Pr1Ra1_VDPM_MRF.png}
  \end{subfigure} 
  ~
  \begin{subfigure}[b]{0.2 \textwidth}
   \centering
   \Large{DRC tAD}
  \includegraphics[width=\textwidth,trim=0 0 0 20,clip]{\voxFld/selected_resfiles/drcAD/atrophyExtent24_drcThInitk-meansCl3Pr1Ra1_VDPM_MRFAD.png}
  \end{subfigure}
  \vspace{0.5em}

  \begin{subfigure}[b]{0.2 \textwidth}
   \centering
   \Large{DRC PCA}
  \includegraphics[width=\textwidth,trim=0 0 0 20,clip]{\voxFld/selected_resfiles/drcPCA/atrophyExtent24_drcThInitk-meansCl5Pr1Ra1_VDPM_MRFPCA.png}
  \end{subfigure}
  ~
  \begin{subfigure}[b]{0.2 \textwidth}
   \centering
   \Large{ADNI PET AV45}
  \includegraphics[width=\textwidth,trim=0 0 0 20,clip]{\voxFld/selected_resfiles/adniPet/atrophyExtent24_adniPetInitk-meansCl18Pr1Ra1_VDPM_MRF.png}
  \end{subfigure}
  
  \small{Marinescu et al., NeuroImage, 2019}\\
  %\small{source code: https://github.com/mrazvan22/dive}

\end{figure}


\end{frame}



% \begin{frame}
% \frametitle{DIVE Estimates the Temporal Evolution of Pathology, Enabling Understanding of Disease Mechanisms}
% 
% 
% \newcommand{\scalingFactor}{1.2}
% \newcommand{\gradLimLeft}{-1.6}
% \newcommand{\gradLimRight}{1.6}
% 
% \definecolor{barGreen}{rgb}{0.4,1,0.4}
% 
% \newcommand{\speed}{2}
% 
% \vfill
% 
% 
% % FWHM0 avg thickness map MCI & AD
% \begin{figure}[h]
%   \centering
%   \vspace{-1em}
% 
%   % do the legend colorbar
%   \begin{subfigure}[b]{0.45\textwidth}
%    \centering
%   \begin{tikzpicture}[scale=\scalingFactor]
%     \shade[left color=red,right color=yellow] (\gradLimLeft,2.5) rectangle (-0.8,2.75);
%     \shade[left color=yellow,right color=barGreen] (-0.8,2.5) rectangle (0,2.75);
%     \shade[left color=barGreen,right color=cyan] (0,2.5) rectangle (0.8,2.75);	
%     \shade[left color=cyan,right color=blue] (0.8,2.5) rectangle (\gradLimRight,2.75);   
% 
%     \node[inner sep=0] (corr_text) at (\gradLimLeft,3) {severe pathology};
%     \node[inner sep=0] (corr_text) at (\gradLimRight,3) {moderate pathology};
%   \end{tikzpicture}
%   \vspace{0.3em}
%   \end{subfigure}
%   
% 
% 	\begin{animateinline}[autoplay,loop]{\speed}  
%    \multiframe{20}{i=10+1}{% loop through pictures
% %  \multiframe{1}{i=10+1}{% loop through pictures \multiframe{nrOfPics}{i=initialVal+increment} 
%   
%   \parbox{\textwidth}{
%   \centering
%   
%   \begin{subfigure}[b]{0.25 \textwidth}
%    \centering
%    \Large{ADNI MRI}
%     \includegraphics[width=\textwidth]{\voxFld/selected_resfiles/adniThick/movie_adniThInitk-meansCl3Pr1Ra1_VDPM_MRFtext_\i}
%   \end{subfigure} 
%   ~
%   \begin{subfigure}[b]{0.25 \textwidth}
%    \centering
%    \Large{DRC tAD}
%   \includegraphics[width=\textwidth]{\voxFld/selected_resfiles/drcAD/movie_drcThInitk-meansCl3Pr1Ra1_VDPM_MRFADtext_\i}
%   \end{subfigure}
%   \vspace{0.5em}
% 
% 
%   \begin{subfigure}[b]{0.25 \textwidth}
%    \centering
%    \Large{DRC PCA}
%     \includegraphics[width=\textwidth]{\voxFld/selected_resfiles/drcPCA/movie_drcThInitk-meansCl5Pr1Ra1_VDPM_MRFPCAtext_\i} 
%   \end{subfigure}
%   ~
%   \begin{subfigure}[b]{0.25 \textwidth}
%    \centering
%    \Large{ADNI PET AV45}
%    \includegraphics[width=\textwidth]{\voxFld/selected_resfiles/adniPet/movie_adniPetInitk-meansCl18Pr1Ra1_VDPM_MRFtext_\i} 
%   \end{subfigure}
%   
%   }  
%   }
%   \end{animateinline}
% 
%   
%   \small{Marinescu et al., Neuroimage, 2019}\\
%   \small{source code: https://github.com/mrazvan22/dive}
% 
% \end{figure}
% 
% % \begin{itemize}
% %  \item Animations generated using BrainPainter: https://github.com/mrazvan22/brain-coloring
% % \end{itemize}
% 
% \end{frame}




%%%%%%%%%%%%%%%%%%%%%%%%%%%%%%%%%%%%%%%%%%%%%%%%%%%%%%%%%%%%%

% \newcommand{\ipmiPaperFold}{.}
\newcommand{\outFoldADNICVbrains}{../overview/figures_ipmi_paper/crossvalid/adniThavgFWHM0Initk-meansCl3Pr0Ra1_VWDPMMean}
\newcommand{\scaleFig}{0.19}

\begin{frame}
\frametitle{Validation - Model Robustly Estimates Atrophy Patterns}

\textbf{Method:} Tested the consistency of the spatial clustering in ADNI using 10-fold CV\\
\vspace{1em}

\textbf{Results:} Good agreement in terms of spatial distribution (dice score 0.89)\\

\begin{figure}[h]
    \centering
    
%     \newcounter{classnumber}
    \foreach \n in {1,...,5}{
    \begin{subfigure}[b]{\scaleFig\textwidth}
    \centering
    f=\n \\
    \includegraphics[width=\textwidth]{\outFoldADNICVbrains/blend\n.png}\\
%     \vspace{-1.5em}
    \includegraphics[width=\textwidth,trim=0 0 0 25,clip]{../overview/figures/cogCorr/trajSamplesOneFig_cogCorr_adniThFWHM0Initk-meansCl3Pr0Ra1Mrf5_VWDPMMean_f\n.png}
    \end{subfigure}
    }
    
%     \caption{Cross-validation}
    \label{fig:ADNICVbrains}
\end{figure}
\vspace{-2em}
\begin{center}
\small{Marinescu et al., Neuroimage, 2019}\\
%\small{source code: https://github.com/mrazvan22/dive}
\end{center}

\end{frame}

%%%%%%%%%%%%%%%%%%%%%%%%%%%%%%%%%%%%%%%%%%%%%%%%%%%%%%%%%%%%%

% \begin{frame}
% \frametitle{Estimated Subject Progression Scores are Clinically Relevant}
% 
% \vspace{-2em}
% 
% \textbf{Hypothesis}: 
% \begin{itemize}
%  \item Clinical relevance $\rightarrow$ DPS correlates with other markers of disease progression
% \end{itemize}
% 
% \vfill
% 
% \textbf{Method}: Ran our model on ADNI using 10-fold cross-validation
% 
% \vfill
% 
% \textbf{Results}: Progression scores correlate well with cognitive tests:
% 
% 
% 
% \newcommand{\figFont}{\small}
% \newcommand{\pValFont}{\tiny}
% 
% \begin{figure}[h]
%   \begin{subfigure}{0.22\textwidth}
%     \centering
%     \figFont{CDRSOB}\\ \pValFont{($\rho = 0.41$, $p < 1e-66$)}
%     \includegraphics[width=1.1\textwidth]{../overview/figures/stagingCogTestsScatterPlot_adniThFWHM0Initk-meansCl3Pr0Ra1Mrf5_VWDPMMean_ADAS13.png}
%   \end{subfigure}
%   \begin{subfigure}{0.22\textwidth}
%     \centering
%     \figFont{ADAS-COG}\\ \pValFont{($\rho = -0.40$, $p < 1e-62$)}
%     \includegraphics[width=1.1\textwidth]{../overview/figures/stagingCogTestsScatterPlot_adniThFWHM0Initk-meansCl3Pr0Ra1Mrf5_VWDPMMean_MMSE.png}
%   \end{subfigure}
%     \begin{subfigure}{0.22\textwidth}
%     \centering
%     \figFont{MMSE}\\ \pValFont{($\rho = -0.39$, $p < 1e-58$)}
%     \includegraphics[width=1.1\textwidth]{../overview/figures/stagingCogTestsScatterPlot_adniThFWHM0Initk-meansCl3Pr0Ra1Mrf5_VWDPMMean_RAVLT.png}
%   \end{subfigure}
%     \begin{subfigure}{0.22\textwidth}
%     \centering
%     \figFont{RAVLT}\\ \pValFont{($\rho = 0.39$, $p < 1e-58$)}
%     \includegraphics[width=1.1\textwidth]{../overview/figures/stagingCogTestsScatterPlot_adniThFWHM0Initk-meansCl3Pr0Ra1Mrf5_VWDPMMean_CDRSOB.png}
%   \end{subfigure}
% \end{figure}
% 
% \vspace{-2em}
% 
% 
% \end{frame}


\begin{frame}{Disease Progression Modelling -- Summary}

\begin{itemize}
 \item We modelled the continuous progression of Alzheimer's disease and related dementias
 
 \vo
 
 \item Used generative bayesian model that does not require labels (unsupervised)
 
 \vo
 
 \item However, such models require good quality data, to perform segmentation and extract disease markers
 
 \vo
 
 \item How can we do such modelling for scans with limited resolution and contrast?
\end{itemize}

 
\end{frame}


% \begin{frame}[label=current]
% \frametitle{Future research directions}
% 
% \textbf{Modelling}:
% \begin{itemize}
% \item Incorporate biological mechanisms (Raj et al., 2012,  Georgiadis et al., 2018)
% \item Incorporate other sources of data: e.g. genetics (Sclesi et al, Brain, 2018)
% \item Account for heterogeneity (e.g. Young et al., Nature Comm., 2018)
% \end{itemize}
% 
% \vfill
% 
% \textbf{Simulations}: 
% \begin{itemize}
% \item Use disease progression models to simulate cohorts (Koval et al., arXiv, 2019)
% \end{itemize}
% 
% \vfill
% 
% \textbf{Applications}:
% \begin{itemize}
% \item Other NDs: Multiple Sclerosis, Huntington's, Parkinson's
% \item Other pathologies: e.g. tumours, lesions
% \end{itemize}
% 
% \end{frame}


% \begin{frame}[label=current]
% \frametitle{BrainPainter features}
% 
% Brain types:
% \begin{figure}
% \centering
% 
% pial
% 
% 
% inflated
% 
% \end{figure}
% 
% 
% \end{frame}


% \begin{frame}[label=current]
% \frametitle{Acknowledgements}
% 
% % \includegraphics[width=\textwidth,trim=0 0 0 150, clip]{cmic_away_day}
%  
%  \begin{small}
% \begin{columns}[T]
% %     \hspace{-4em}
%     \begin{column}{.3\textwidth}
%     \textbf{Collaborators}
%     \begin{enumerate}
%     \item Leon Aksman
%     \item Maura Bellio
%      \item Arman Eshaghi
%      \item Nicholas Firth
%      \item Sara Garbarino
%      \item Marco Lorenzi
%      \item Kyriaki Mengoudi
%      \item Neil Oxtoby
%      \item Alexandra Young
%     \end{enumerate}
% %     \includegraphics[scale=1]{pondLogo.png}  
%     \end{column}
%     
%     
% %     \begin{column}{.6\textwidth}
% %     \begin{center}
% %     \textbf{Project Supervisors}
% %     \end{center}
% %     \vspace{-2em}
% %       \begin{figure}
% %       \begin{subfigure}{0.3\textwidth}
% %       \centering
% %       Daniel Alexander
% %       \includegraphics[height=2cm]{Danny-Alexander.jpeg}  
% %       \end{subfigure}
% % 	\begin{subfigure}{0.3\textwidth}
% % 	\centering
% %       Sebastian Crutch
% %       \includegraphics[height=2cm]{Seb_Crutch_photo.JPG}  
% %       \end{subfigure}
% %   \begin{subfigure}{0.3\textwidth}
% % 	\centering
% %       Polina Golland
% %       \includegraphics[height=2cm, trim=0 00 0 0,clip]{polina}  
% %       \end{subfigure}
% %       
% %     \begin{center}
% %     \textbf{Funders}
% %     \end{center}
% %     %\vspace{-2em}
% %       \begin{subfigure}{0.3\textwidth}
% %       \centering
% %       \includegraphics[height=1cm]{epsrc_logo}  
% %       \end{subfigure}
% % 	\begin{subfigure}{0.3\textwidth}
% % 	\centering
% %       \includegraphics[height=1cm]{CDTlogo}  
% %       \end{subfigure}
% %      
% %       
% %       \vspace{2em}
% % 
% % \end{figure}
% %     
% % \end{column}
% \end{columns}
%  
% \end{small}
% 
% 
% \end{frame}



\begin{frame}{Outline}

\end{frame}

% \begin{frame}{Aim: image reconstruction using *pre-trained* generator models}

\begin{columns}
 \begin{column}{0.5\textwidth}
  
  \begin{itemize}
   \item Adapt the state-of-the-art StyleGAN2 for medical image generation and reconstruction

   \vspace{2em}
   
   \item Vision: bring this technology to medicine
  \end{itemize}
  
 \vspace{1em}
  
 \incw{mriall}{1} 
  
  
 \end{column} 
 
 \begin{column}{0.5\textwidth}
 \centering
 StyleGAN2 (Karras et al, 2019)
 \inc{stylegan} 
 \end{column}

\end{columns}



\end{frame}

\begin{frame}{Motivation 1: Computational requirements}

\begin{itemize}
 \item State-of-the-art deep learning methods are becoming difficult to train:
 \begin{itemize}
 \item Large computational resources
 \item Large datasets
 \item Long time to converge
 
 \incw{datacenter}{0.7}
\end{itemize}
\vo
 \item Currently few labs/companies have the resources to train them

\vo
 \item Solutions moving forward:
 \begin{itemize}
\item Adapting previously-trained models
\item Combine smaller models into larger ones
\end{itemize}
\end{itemize}

 
\end{frame}

\begin{frame}{Motivation 2: Account for distribution shifts}


\begin{itemize}
 \item Distribution shifts happen all the time:
 \begin{itemize}
 \item Changes in hospital scanners, protocols, software upgrades
 \item Can be continuous: population getting older due to better healthcare
\end{itemize}

\vspace{2em} 
 \item Shifts in both inputs and outputs can result in combinatorial effects!
\end{itemize}
% \vspace{1em}

\begin{center}
\incw{compositionality}{1}
\end{center}
 
\end{frame}


\begin{frame}{Motivation 3: Causality}


\begin{itemize}
 \item Closely follow the data-generating process
 \item The right solution to deal with distribution shifts
 \begin{itemize}
 \item Both for inputs and intermediary variables
\end{itemize}
\end{itemize}


\begin{center}
\incw{causality}{1}
\end{center}
 
\end{frame}


\begin{frame}{Method: We perform image reconstruction by combining two models\\
% \begin{itemize}
\ \ \ \ \ \  1. a pre-trained generator G (StyleGAN2)\\
\ \ \ \ \ \  2. a known forward corruption model $f_1$
% \end{itemize}
}

% We combine:

\vt
\incw{brgm_diagram}{1}
 
\end{frame}


\begin{frame}{Reconstructed image is given by computing the Bayesian maximum a-posteriori (MAP) estimate\\
}

\begin{itemize}
 \item We optimise:
$$ w^* = \argmax_w p(w)p(I|w)$$

\item For uninformative prior $p(w)$ and Gaussian noise model (pixelwise independent), we get:

$$ w^* = \argmin_w || I - f \circ G (w) ||_2^2$$

\item This can be optimised with SGD

\item Once we get $w^*$, the the reconstructed image is $G(w^*)$

\end{itemize}

\begin{center}
\vt
\incw{brgm_diagram_mine}{0.6}
\end{center}
 
\end{frame}

\begin{frame}{Reconstruction in this basic formulation doesn't work}

\begin{itemize}
 \item We started from the original StyleGAN2 inversion
 \item Yet the reconstruction was not good
 \item We reached suitable solutions only after several changes 
\end{itemize}

\begin{center}
\vt
\incw{evol}{1}
\end{center}
 
\end{frame}

\begin{frame}{Results on super-resolution}

\begin{itemize}
 \item We achieve state-of-the-art (SOTA) results on small inputs resolutions 16x16
 \item On larger resolutions (>32x32), we achieve very good results, albeit not SOTA
\end{itemize}

\begin{center}
\vt
\incw{sr}{1}
\end{center}
 
\end{frame}

\begin{frame}{Similar results on super-resolution for medical datasets}

\begin{itemize}
 \item We achieve state-of-the-art (SOTA) results on small inputs resolutions 16x16
 \item On larger resolutions (>32x32), we achieve very good results, albeit not SOTA
\end{itemize}

\begin{center}
\vt
\incw{srmed}{0.7}
\end{center}
 
\end{frame}

\begin{frame}{Inpainting also achieves state-of-the-art results}

\begin{itemize}
 \item Best previous method (SN-PatchGAN, CVPR 2019) does not work for large masks
 \item Our method can ``hypothesize'' missing structure
\end{itemize}

\begin{center}
% \vt
\incw{inpaint2}{0.61}
\end{center}
 
\end{frame}

\begin{frame}{Inpainting also achieves state-of-the-art results}

\begin{itemize}
 \item Best previous method (SN-PatchGAN, CVPR 2019) does not work for large masks
 \item Our method can ``hypothesize'' missing structure
\end{itemize}

\begin{center}
\vt
\incw{inpaint_xray}{0.5}\incw{inpaint_brains}{0.5}
\end{center}
 
\end{frame}

\begin{frame}{Our method also has limitations}

\begin{itemize}
 \item It can fail for images that are too dissimilar to the training ones
 \begin{itemize}
 \item Because generator cannot extrapolate easily
 \end{itemize}
 \begin{center}
 \incw{failure}{0.6} 
 \end{center}
 
 \item Can be inconsistent with the input image
 \begin{center}
 \incw{inconsistency}{0.6}
 \end{center}
 
\end{itemize}
 
\end{frame}


\begin{frame}{Conclusion}
 
 \begin{itemize}
  \item Proposed a method for image reconstruction using pre-trained deep generative models
  \item Solution is given by the Bayesian MAP estimate
  \item State-of-the-art results on super-resolution and inpainting
  
 \end{itemize}

 
\end{frame}






\begin{frame}{Outline}
 
 
\end{frame}

\begin{frame}{Future work}



\begin{columns}[t]
\begin{column}{0.5\textwidth}
\centering

Biological simulators\\
\incw{heartsim}{0.5}

\vo

Multimodal modelling\\
images + text + structural data  
\incw{xray}{0.3}\incw{medreport}{0.267}

\vo

Domain knowledge through large-scale parsing of medical articles\\
\incw{lotsoftext}{0.8}

\end{column}
\begin{column}{0.5\textwidth}
\centering

Better and faster reconstruction of medical images\\
\incw{mrirecon}{0.6}

\vo

Disease Progression Modelling
\incw{daninet}{0.6}

\end{column}
\end{columns}


\end{frame}


\begin{frame}{Future work: Brain tissue and anatomy simulator}


Simulator for brain anatomy from genetics:
\begin{itemize}
 \item Using deep generative models
 \item Accouting for distributions shifts
 \item Following causal principles
\end{itemize}

\vt

\incw{brainsimulator}{0.9}

\end{frame}

\begin{frame}{Long-term vision}



\begin{columns}[t]
\begin{column}{0.5\textwidth}
\centering


Early diagnosis and prognosis
\incw{smartwatch}{0.7}

AI augumenting humans\\
\incw{vr}{0.7}




\end{column}
\begin{column}{0.5\textwidth}
\centering


Robotic Surgery
\incw{roboticsurgery}{0.8}

\vo

Drug development
\incw{drugdevelopment}{0.8}


\end{column}
\end{columns}




\end{frame}


\end{document}



